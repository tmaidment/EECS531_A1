
% Default to the notebook output style

    


% Inherit from the specified cell style.




    
\documentclass[11pt]{article}

    
    
    \usepackage[T1]{fontenc}
    % Nicer default font (+ math font) than Computer Modern for most use cases
    \usepackage{mathpazo}

    % Basic figure setup, for now with no caption control since it's done
    % automatically by Pandoc (which extracts ![](path) syntax from Markdown).
    \usepackage{graphicx}
    % We will generate all images so they have a width \maxwidth. This means
    % that they will get their normal width if they fit onto the page, but
    % are scaled down if they would overflow the margins.
    \makeatletter
    \def\maxwidth{\ifdim\Gin@nat@width>\linewidth\linewidth
    \else\Gin@nat@width\fi}
    \makeatother
    \let\Oldincludegraphics\includegraphics
    % Set max figure width to be 80% of text width, for now hardcoded.
    \renewcommand{\includegraphics}[1]{\Oldincludegraphics[width=.8\maxwidth]{#1}}
    % Ensure that by default, figures have no caption (until we provide a
    % proper Figure object with a Caption API and a way to capture that
    % in the conversion process - todo).
    \usepackage{caption}
    \DeclareCaptionLabelFormat{nolabel}{}
    \captionsetup{labelformat=nolabel}

    \usepackage{adjustbox} % Used to constrain images to a maximum size 
    \usepackage{xcolor} % Allow colors to be defined
    \usepackage{enumerate} % Needed for markdown enumerations to work
    \usepackage{geometry} % Used to adjust the document margins
    \usepackage{amsmath} % Equations
    \usepackage{amssymb} % Equations
    \usepackage{textcomp} % defines textquotesingle
    % Hack from http://tex.stackexchange.com/a/47451/13684:
    \AtBeginDocument{%
        \def\PYZsq{\textquotesingle}% Upright quotes in Pygmentized code
    }
    \usepackage{upquote} % Upright quotes for verbatim code
    \usepackage{eurosym} % defines \euro
    \usepackage[mathletters]{ucs} % Extended unicode (utf-8) support
    \usepackage[utf8x]{inputenc} % Allow utf-8 characters in the tex document
    \usepackage{fancyvrb} % verbatim replacement that allows latex
    \usepackage{grffile} % extends the file name processing of package graphics 
                         % to support a larger range 
    % The hyperref package gives us a pdf with properly built
    % internal navigation ('pdf bookmarks' for the table of contents,
    % internal cross-reference links, web links for URLs, etc.)
    \usepackage{hyperref}
    \usepackage{longtable} % longtable support required by pandoc >1.10
    \usepackage{booktabs}  % table support for pandoc > 1.12.2
    \usepackage[inline]{enumitem} % IRkernel/repr support (it uses the enumerate* environment)
    \usepackage[normalem]{ulem} % ulem is needed to support strikethroughs (\sout)
                                % normalem makes italics be italics, not underlines
    

    
    
    % Colors for the hyperref package
    \definecolor{urlcolor}{rgb}{0,.145,.698}
    \definecolor{linkcolor}{rgb}{.71,0.21,0.01}
    \definecolor{citecolor}{rgb}{.12,.54,.11}

    % ANSI colors
    \definecolor{ansi-black}{HTML}{3E424D}
    \definecolor{ansi-black-intense}{HTML}{282C36}
    \definecolor{ansi-red}{HTML}{E75C58}
    \definecolor{ansi-red-intense}{HTML}{B22B31}
    \definecolor{ansi-green}{HTML}{00A250}
    \definecolor{ansi-green-intense}{HTML}{007427}
    \definecolor{ansi-yellow}{HTML}{DDB62B}
    \definecolor{ansi-yellow-intense}{HTML}{B27D12}
    \definecolor{ansi-blue}{HTML}{208FFB}
    \definecolor{ansi-blue-intense}{HTML}{0065CA}
    \definecolor{ansi-magenta}{HTML}{D160C4}
    \definecolor{ansi-magenta-intense}{HTML}{A03196}
    \definecolor{ansi-cyan}{HTML}{60C6C8}
    \definecolor{ansi-cyan-intense}{HTML}{258F8F}
    \definecolor{ansi-white}{HTML}{C5C1B4}
    \definecolor{ansi-white-intense}{HTML}{A1A6B2}

    % commands and environments needed by pandoc snippets
    % extracted from the output of `pandoc -s`
    \providecommand{\tightlist}{%
      \setlength{\itemsep}{0pt}\setlength{\parskip}{0pt}}
    \DefineVerbatimEnvironment{Highlighting}{Verbatim}{commandchars=\\\{\}}
    % Add ',fontsize=\small' for more characters per line
    \newenvironment{Shaded}{}{}
    \newcommand{\KeywordTok}[1]{\textcolor[rgb]{0.00,0.44,0.13}{\textbf{{#1}}}}
    \newcommand{\DataTypeTok}[1]{\textcolor[rgb]{0.56,0.13,0.00}{{#1}}}
    \newcommand{\DecValTok}[1]{\textcolor[rgb]{0.25,0.63,0.44}{{#1}}}
    \newcommand{\BaseNTok}[1]{\textcolor[rgb]{0.25,0.63,0.44}{{#1}}}
    \newcommand{\FloatTok}[1]{\textcolor[rgb]{0.25,0.63,0.44}{{#1}}}
    \newcommand{\CharTok}[1]{\textcolor[rgb]{0.25,0.44,0.63}{{#1}}}
    \newcommand{\StringTok}[1]{\textcolor[rgb]{0.25,0.44,0.63}{{#1}}}
    \newcommand{\CommentTok}[1]{\textcolor[rgb]{0.38,0.63,0.69}{\textit{{#1}}}}
    \newcommand{\OtherTok}[1]{\textcolor[rgb]{0.00,0.44,0.13}{{#1}}}
    \newcommand{\AlertTok}[1]{\textcolor[rgb]{1.00,0.00,0.00}{\textbf{{#1}}}}
    \newcommand{\FunctionTok}[1]{\textcolor[rgb]{0.02,0.16,0.49}{{#1}}}
    \newcommand{\RegionMarkerTok}[1]{{#1}}
    \newcommand{\ErrorTok}[1]{\textcolor[rgb]{1.00,0.00,0.00}{\textbf{{#1}}}}
    \newcommand{\NormalTok}[1]{{#1}}
    
    % Additional commands for more recent versions of Pandoc
    \newcommand{\ConstantTok}[1]{\textcolor[rgb]{0.53,0.00,0.00}{{#1}}}
    \newcommand{\SpecialCharTok}[1]{\textcolor[rgb]{0.25,0.44,0.63}{{#1}}}
    \newcommand{\VerbatimStringTok}[1]{\textcolor[rgb]{0.25,0.44,0.63}{{#1}}}
    \newcommand{\SpecialStringTok}[1]{\textcolor[rgb]{0.73,0.40,0.53}{{#1}}}
    \newcommand{\ImportTok}[1]{{#1}}
    \newcommand{\DocumentationTok}[1]{\textcolor[rgb]{0.73,0.13,0.13}{\textit{{#1}}}}
    \newcommand{\AnnotationTok}[1]{\textcolor[rgb]{0.38,0.63,0.69}{\textbf{\textit{{#1}}}}}
    \newcommand{\CommentVarTok}[1]{\textcolor[rgb]{0.38,0.63,0.69}{\textbf{\textit{{#1}}}}}
    \newcommand{\VariableTok}[1]{\textcolor[rgb]{0.10,0.09,0.49}{{#1}}}
    \newcommand{\ControlFlowTok}[1]{\textcolor[rgb]{0.00,0.44,0.13}{\textbf{{#1}}}}
    \newcommand{\OperatorTok}[1]{\textcolor[rgb]{0.40,0.40,0.40}{{#1}}}
    \newcommand{\BuiltInTok}[1]{{#1}}
    \newcommand{\ExtensionTok}[1]{{#1}}
    \newcommand{\PreprocessorTok}[1]{\textcolor[rgb]{0.74,0.48,0.00}{{#1}}}
    \newcommand{\AttributeTok}[1]{\textcolor[rgb]{0.49,0.56,0.16}{{#1}}}
    \newcommand{\InformationTok}[1]{\textcolor[rgb]{0.38,0.63,0.69}{\textbf{\textit{{#1}}}}}
    \newcommand{\WarningTok}[1]{\textcolor[rgb]{0.38,0.63,0.69}{\textbf{\textit{{#1}}}}}
    
    
    % Define a nice break command that doesn't care if a line doesn't already
    % exist.
    \def\br{\hspace*{\fill} \\* }
    % Math Jax compatability definitions
    \def\gt{>}
    \def\lt{<}
    % Document parameters
    \title{A1\_tdm47\_E4}
    
    
    

    % Pygments definitions
    
\makeatletter
\def\PY@reset{\let\PY@it=\relax \let\PY@bf=\relax%
    \let\PY@ul=\relax \let\PY@tc=\relax%
    \let\PY@bc=\relax \let\PY@ff=\relax}
\def\PY@tok#1{\csname PY@tok@#1\endcsname}
\def\PY@toks#1+{\ifx\relax#1\empty\else%
    \PY@tok{#1}\expandafter\PY@toks\fi}
\def\PY@do#1{\PY@bc{\PY@tc{\PY@ul{%
    \PY@it{\PY@bf{\PY@ff{#1}}}}}}}
\def\PY#1#2{\PY@reset\PY@toks#1+\relax+\PY@do{#2}}

\expandafter\def\csname PY@tok@w\endcsname{\def\PY@tc##1{\textcolor[rgb]{0.73,0.73,0.73}{##1}}}
\expandafter\def\csname PY@tok@c\endcsname{\let\PY@it=\textit\def\PY@tc##1{\textcolor[rgb]{0.25,0.50,0.50}{##1}}}
\expandafter\def\csname PY@tok@cp\endcsname{\def\PY@tc##1{\textcolor[rgb]{0.74,0.48,0.00}{##1}}}
\expandafter\def\csname PY@tok@k\endcsname{\let\PY@bf=\textbf\def\PY@tc##1{\textcolor[rgb]{0.00,0.50,0.00}{##1}}}
\expandafter\def\csname PY@tok@kp\endcsname{\def\PY@tc##1{\textcolor[rgb]{0.00,0.50,0.00}{##1}}}
\expandafter\def\csname PY@tok@kt\endcsname{\def\PY@tc##1{\textcolor[rgb]{0.69,0.00,0.25}{##1}}}
\expandafter\def\csname PY@tok@o\endcsname{\def\PY@tc##1{\textcolor[rgb]{0.40,0.40,0.40}{##1}}}
\expandafter\def\csname PY@tok@ow\endcsname{\let\PY@bf=\textbf\def\PY@tc##1{\textcolor[rgb]{0.67,0.13,1.00}{##1}}}
\expandafter\def\csname PY@tok@nb\endcsname{\def\PY@tc##1{\textcolor[rgb]{0.00,0.50,0.00}{##1}}}
\expandafter\def\csname PY@tok@nf\endcsname{\def\PY@tc##1{\textcolor[rgb]{0.00,0.00,1.00}{##1}}}
\expandafter\def\csname PY@tok@nc\endcsname{\let\PY@bf=\textbf\def\PY@tc##1{\textcolor[rgb]{0.00,0.00,1.00}{##1}}}
\expandafter\def\csname PY@tok@nn\endcsname{\let\PY@bf=\textbf\def\PY@tc##1{\textcolor[rgb]{0.00,0.00,1.00}{##1}}}
\expandafter\def\csname PY@tok@ne\endcsname{\let\PY@bf=\textbf\def\PY@tc##1{\textcolor[rgb]{0.82,0.25,0.23}{##1}}}
\expandafter\def\csname PY@tok@nv\endcsname{\def\PY@tc##1{\textcolor[rgb]{0.10,0.09,0.49}{##1}}}
\expandafter\def\csname PY@tok@no\endcsname{\def\PY@tc##1{\textcolor[rgb]{0.53,0.00,0.00}{##1}}}
\expandafter\def\csname PY@tok@nl\endcsname{\def\PY@tc##1{\textcolor[rgb]{0.63,0.63,0.00}{##1}}}
\expandafter\def\csname PY@tok@ni\endcsname{\let\PY@bf=\textbf\def\PY@tc##1{\textcolor[rgb]{0.60,0.60,0.60}{##1}}}
\expandafter\def\csname PY@tok@na\endcsname{\def\PY@tc##1{\textcolor[rgb]{0.49,0.56,0.16}{##1}}}
\expandafter\def\csname PY@tok@nt\endcsname{\let\PY@bf=\textbf\def\PY@tc##1{\textcolor[rgb]{0.00,0.50,0.00}{##1}}}
\expandafter\def\csname PY@tok@nd\endcsname{\def\PY@tc##1{\textcolor[rgb]{0.67,0.13,1.00}{##1}}}
\expandafter\def\csname PY@tok@s\endcsname{\def\PY@tc##1{\textcolor[rgb]{0.73,0.13,0.13}{##1}}}
\expandafter\def\csname PY@tok@sd\endcsname{\let\PY@it=\textit\def\PY@tc##1{\textcolor[rgb]{0.73,0.13,0.13}{##1}}}
\expandafter\def\csname PY@tok@si\endcsname{\let\PY@bf=\textbf\def\PY@tc##1{\textcolor[rgb]{0.73,0.40,0.53}{##1}}}
\expandafter\def\csname PY@tok@se\endcsname{\let\PY@bf=\textbf\def\PY@tc##1{\textcolor[rgb]{0.73,0.40,0.13}{##1}}}
\expandafter\def\csname PY@tok@sr\endcsname{\def\PY@tc##1{\textcolor[rgb]{0.73,0.40,0.53}{##1}}}
\expandafter\def\csname PY@tok@ss\endcsname{\def\PY@tc##1{\textcolor[rgb]{0.10,0.09,0.49}{##1}}}
\expandafter\def\csname PY@tok@sx\endcsname{\def\PY@tc##1{\textcolor[rgb]{0.00,0.50,0.00}{##1}}}
\expandafter\def\csname PY@tok@m\endcsname{\def\PY@tc##1{\textcolor[rgb]{0.40,0.40,0.40}{##1}}}
\expandafter\def\csname PY@tok@gh\endcsname{\let\PY@bf=\textbf\def\PY@tc##1{\textcolor[rgb]{0.00,0.00,0.50}{##1}}}
\expandafter\def\csname PY@tok@gu\endcsname{\let\PY@bf=\textbf\def\PY@tc##1{\textcolor[rgb]{0.50,0.00,0.50}{##1}}}
\expandafter\def\csname PY@tok@gd\endcsname{\def\PY@tc##1{\textcolor[rgb]{0.63,0.00,0.00}{##1}}}
\expandafter\def\csname PY@tok@gi\endcsname{\def\PY@tc##1{\textcolor[rgb]{0.00,0.63,0.00}{##1}}}
\expandafter\def\csname PY@tok@gr\endcsname{\def\PY@tc##1{\textcolor[rgb]{1.00,0.00,0.00}{##1}}}
\expandafter\def\csname PY@tok@ge\endcsname{\let\PY@it=\textit}
\expandafter\def\csname PY@tok@gs\endcsname{\let\PY@bf=\textbf}
\expandafter\def\csname PY@tok@gp\endcsname{\let\PY@bf=\textbf\def\PY@tc##1{\textcolor[rgb]{0.00,0.00,0.50}{##1}}}
\expandafter\def\csname PY@tok@go\endcsname{\def\PY@tc##1{\textcolor[rgb]{0.53,0.53,0.53}{##1}}}
\expandafter\def\csname PY@tok@gt\endcsname{\def\PY@tc##1{\textcolor[rgb]{0.00,0.27,0.87}{##1}}}
\expandafter\def\csname PY@tok@err\endcsname{\def\PY@bc##1{\setlength{\fboxsep}{0pt}\fcolorbox[rgb]{1.00,0.00,0.00}{1,1,1}{\strut ##1}}}
\expandafter\def\csname PY@tok@kc\endcsname{\let\PY@bf=\textbf\def\PY@tc##1{\textcolor[rgb]{0.00,0.50,0.00}{##1}}}
\expandafter\def\csname PY@tok@kd\endcsname{\let\PY@bf=\textbf\def\PY@tc##1{\textcolor[rgb]{0.00,0.50,0.00}{##1}}}
\expandafter\def\csname PY@tok@kn\endcsname{\let\PY@bf=\textbf\def\PY@tc##1{\textcolor[rgb]{0.00,0.50,0.00}{##1}}}
\expandafter\def\csname PY@tok@kr\endcsname{\let\PY@bf=\textbf\def\PY@tc##1{\textcolor[rgb]{0.00,0.50,0.00}{##1}}}
\expandafter\def\csname PY@tok@bp\endcsname{\def\PY@tc##1{\textcolor[rgb]{0.00,0.50,0.00}{##1}}}
\expandafter\def\csname PY@tok@fm\endcsname{\def\PY@tc##1{\textcolor[rgb]{0.00,0.00,1.00}{##1}}}
\expandafter\def\csname PY@tok@vc\endcsname{\def\PY@tc##1{\textcolor[rgb]{0.10,0.09,0.49}{##1}}}
\expandafter\def\csname PY@tok@vg\endcsname{\def\PY@tc##1{\textcolor[rgb]{0.10,0.09,0.49}{##1}}}
\expandafter\def\csname PY@tok@vi\endcsname{\def\PY@tc##1{\textcolor[rgb]{0.10,0.09,0.49}{##1}}}
\expandafter\def\csname PY@tok@vm\endcsname{\def\PY@tc##1{\textcolor[rgb]{0.10,0.09,0.49}{##1}}}
\expandafter\def\csname PY@tok@sa\endcsname{\def\PY@tc##1{\textcolor[rgb]{0.73,0.13,0.13}{##1}}}
\expandafter\def\csname PY@tok@sb\endcsname{\def\PY@tc##1{\textcolor[rgb]{0.73,0.13,0.13}{##1}}}
\expandafter\def\csname PY@tok@sc\endcsname{\def\PY@tc##1{\textcolor[rgb]{0.73,0.13,0.13}{##1}}}
\expandafter\def\csname PY@tok@dl\endcsname{\def\PY@tc##1{\textcolor[rgb]{0.73,0.13,0.13}{##1}}}
\expandafter\def\csname PY@tok@s2\endcsname{\def\PY@tc##1{\textcolor[rgb]{0.73,0.13,0.13}{##1}}}
\expandafter\def\csname PY@tok@sh\endcsname{\def\PY@tc##1{\textcolor[rgb]{0.73,0.13,0.13}{##1}}}
\expandafter\def\csname PY@tok@s1\endcsname{\def\PY@tc##1{\textcolor[rgb]{0.73,0.13,0.13}{##1}}}
\expandafter\def\csname PY@tok@mb\endcsname{\def\PY@tc##1{\textcolor[rgb]{0.40,0.40,0.40}{##1}}}
\expandafter\def\csname PY@tok@mf\endcsname{\def\PY@tc##1{\textcolor[rgb]{0.40,0.40,0.40}{##1}}}
\expandafter\def\csname PY@tok@mh\endcsname{\def\PY@tc##1{\textcolor[rgb]{0.40,0.40,0.40}{##1}}}
\expandafter\def\csname PY@tok@mi\endcsname{\def\PY@tc##1{\textcolor[rgb]{0.40,0.40,0.40}{##1}}}
\expandafter\def\csname PY@tok@il\endcsname{\def\PY@tc##1{\textcolor[rgb]{0.40,0.40,0.40}{##1}}}
\expandafter\def\csname PY@tok@mo\endcsname{\def\PY@tc##1{\textcolor[rgb]{0.40,0.40,0.40}{##1}}}
\expandafter\def\csname PY@tok@ch\endcsname{\let\PY@it=\textit\def\PY@tc##1{\textcolor[rgb]{0.25,0.50,0.50}{##1}}}
\expandafter\def\csname PY@tok@cm\endcsname{\let\PY@it=\textit\def\PY@tc##1{\textcolor[rgb]{0.25,0.50,0.50}{##1}}}
\expandafter\def\csname PY@tok@cpf\endcsname{\let\PY@it=\textit\def\PY@tc##1{\textcolor[rgb]{0.25,0.50,0.50}{##1}}}
\expandafter\def\csname PY@tok@c1\endcsname{\let\PY@it=\textit\def\PY@tc##1{\textcolor[rgb]{0.25,0.50,0.50}{##1}}}
\expandafter\def\csname PY@tok@cs\endcsname{\let\PY@it=\textit\def\PY@tc##1{\textcolor[rgb]{0.25,0.50,0.50}{##1}}}

\def\PYZbs{\char`\\}
\def\PYZus{\char`\_}
\def\PYZob{\char`\{}
\def\PYZcb{\char`\}}
\def\PYZca{\char`\^}
\def\PYZam{\char`\&}
\def\PYZlt{\char`\<}
\def\PYZgt{\char`\>}
\def\PYZsh{\char`\#}
\def\PYZpc{\char`\%}
\def\PYZdl{\char`\$}
\def\PYZhy{\char`\-}
\def\PYZsq{\char`\'}
\def\PYZdq{\char`\"}
\def\PYZti{\char`\~}
% for compatibility with earlier versions
\def\PYZat{@}
\def\PYZlb{[}
\def\PYZrb{]}
\makeatother


    % Exact colors from NB
    \definecolor{incolor}{rgb}{0.0, 0.0, 0.5}
    \definecolor{outcolor}{rgb}{0.545, 0.0, 0.0}



    
    % Prevent overflowing lines due to hard-to-break entities
    \sloppy 
    % Setup hyperref package
    \hypersetup{
      breaklinks=true,  % so long urls are correctly broken across lines
      colorlinks=true,
      urlcolor=urlcolor,
      linkcolor=linkcolor,
      citecolor=citecolor,
      }
    % Slightly bigger margins than the latex defaults
    
    \geometry{verbose,tmargin=1in,bmargin=1in,lmargin=1in,rmargin=1in}
    
    

    \begin{document}
    
    
    \maketitle
    
    

    
    \hypertarget{eecs-531---a1}{%
\section{EECS 531 - A1}\label{eecs-531---a1}}

\hypertarget{tristan-maidment-tdm47}{%
\subsubsection{Tristan Maidment (tdm47)}\label{tristan-maidment-tdm47}}

\hypertarget{assignment-1}{%
\subsubsection{Assignment 1}\label{assignment-1}}

\hypertarget{exercise-4}{%
\paragraph{Exercise 4}\label{exercise-4}}

    To determine the ROC curve for the template matching method in Exercise
3, I first need to calculate the True Positive Rate and False Positive
Rate, as defined by the lecture slides.

The first step is to import all of the code needed to do feature
detection from the previous exercise. This contains all the methods
used, and will be used to calculate the different ROC curves with
different values of noise added.

    \begin{Verbatim}[commandchars=\\\{\}]
{\color{incolor}In [{\color{incolor}16}]:} \PY{o}{\PYZpc{}}\PY{k}{matplotlib} inline
         \PY{k+kn}{import} \PY{n+nn}{cv2}
         \PY{k+kn}{import} \PY{n+nn}{math}
         \PY{k+kn}{from} \PY{n+nn}{matplotlib} \PY{k}{import} \PY{n}{pyplot} \PY{k}{as} \PY{n}{plt}
         \PY{k+kn}{from} \PY{n+nn}{matplotlib} \PY{k}{import} \PY{n}{patches} \PY{k}{as} \PY{n}{patches}
         \PY{k+kn}{import} \PY{n+nn}{numpy} \PY{k}{as} \PY{n+nn}{np}
         
         \PY{k}{def} \PY{n+nf}{correlation}\PY{p}{(}\PY{n}{img}\PY{p}{,} \PY{n}{kernel}\PY{p}{)}\PY{p}{:}
             \PY{n}{width}\PY{p}{,} \PY{n}{height}\PY{p}{,} \PY{n}{channels} \PY{o}{=} \PY{n}{img}\PY{o}{.}\PY{n}{shape} 
             \PY{n}{k\PYZus{}width}\PY{p}{,} \PY{n}{k\PYZus{}height} \PY{o}{=} \PY{n}{kernel}\PY{o}{.}\PY{n}{shape}
             
             \PY{n}{k\PYZus{}half\PYZus{}width} \PY{o}{=} \PY{n}{math}\PY{o}{.}\PY{n}{floor}\PY{p}{(}\PY{n}{k\PYZus{}width}\PY{o}{/}\PY{l+m+mi}{2}\PY{p}{)}
             \PY{n}{k\PYZus{}half\PYZus{}width\PYZus{}2} \PY{o}{=} \PY{n}{math}\PY{o}{.}\PY{n}{ceil}\PY{p}{(}\PY{n}{k\PYZus{}width}\PY{o}{/}\PY{l+m+mi}{2}\PY{p}{)}
             \PY{n}{k\PYZus{}half\PYZus{}height} \PY{o}{=} \PY{n}{math}\PY{o}{.}\PY{n}{floor}\PY{p}{(}\PY{n}{k\PYZus{}height}\PY{o}{/}\PY{l+m+mi}{2}\PY{p}{)}
             \PY{n}{k\PYZus{}half\PYZus{}height\PYZus{}2} \PY{o}{=} \PY{n}{math}\PY{o}{.}\PY{n}{ceil}\PY{p}{(}\PY{n}{k\PYZus{}height}\PY{o}{/}\PY{l+m+mi}{2}\PY{p}{)}
                 
             \PY{n}{ret} \PY{o}{=} \PY{n}{np}\PY{o}{.}\PY{n}{zeros}\PY{p}{(}\PY{n}{img}\PY{o}{.}\PY{n}{shape}\PY{p}{)} 
             
             \PY{n}{img\PYZus{}pad} \PY{o}{=} \PY{n}{np}\PY{o}{.}\PY{n}{ones}\PY{p}{(}\PY{p}{(}\PY{n}{width} \PY{o}{+} \PY{n}{k\PYZus{}width} \PY{o}{+} \PY{l+m+mi}{1} \PY{p}{,} \PY{n}{height} \PY{o}{+} \PY{n}{k\PYZus{}width} \PY{o}{+} \PY{l+m+mi}{1}\PY{p}{,} \PY{n}{channels}\PY{p}{)}\PY{p}{)}
             \PY{n}{img\PYZus{}pad}\PY{p}{[}\PY{n}{k\PYZus{}half\PYZus{}width} \PY{p}{:} \PY{o}{\PYZhy{}}\PY{n}{k\PYZus{}half\PYZus{}width\PYZus{}2} \PY{o}{\PYZhy{}} \PY{l+m+mi}{1}\PY{p}{,}
                     \PY{n}{k\PYZus{}half\PYZus{}height} \PY{p}{:} \PY{o}{\PYZhy{}}\PY{n}{k\PYZus{}half\PYZus{}height\PYZus{}2} \PY{o}{\PYZhy{}} \PY{l+m+mi}{1}\PY{p}{]} \PY{o}{=} \PY{n}{img} 
             
             \PY{k}{for} \PY{n}{x} \PY{o+ow}{in} \PY{n+nb}{range}\PY{p}{(}\PY{n}{width}\PY{p}{)}\PY{p}{:}
                 \PY{k}{for} \PY{n}{y} \PY{o+ow}{in} \PY{n+nb}{range}\PY{p}{(}\PY{n}{height}\PY{p}{)}\PY{p}{:}
                     \PY{k}{for} \PY{n}{c} \PY{o+ow}{in} \PY{n+nb}{range}\PY{p}{(}\PY{n}{channels}\PY{p}{)}\PY{p}{:}
                         \PY{n}{ret}\PY{p}{[}\PY{n}{x}\PY{p}{,} \PY{n}{y}\PY{p}{,} \PY{n}{c}\PY{p}{]} \PY{o}{=} \PY{n}{np}\PY{o}{.}\PY{n}{power}\PY{p}{(}\PY{n}{np}\PY{o}{.}\PY{n}{subtract}\PY{p}{(}\PY{n}{img\PYZus{}pad}\PY{p}{[}\PY{n}{x}\PY{p}{:} \PY{n}{x} \PY{o}{+} \PY{n}{k\PYZus{}width}\PY{p}{,} 
                                                                     \PY{n}{y}  \PY{p}{:} \PY{n}{y} \PY{o}{+} \PY{n}{k\PYZus{}height}\PY{p}{,}
                                                                     \PY{n}{c}\PY{p}{]}\PY{p}{,} \PY{n}{kernel}\PY{p}{)}\PY{p}{,} \PY{l+m+mi}{2}\PY{p}{)}\PY{o}{.}\PY{n}{sum}\PY{p}{(}\PY{p}{)}
             \PY{k}{return} \PY{n}{ret}\PY{o}{/}\PY{n}{ret}\PY{o}{.}\PY{n}{max}\PY{p}{(}\PY{p}{)}
         
         \PY{k}{def} \PY{n+nf}{create\PYZus{}kernel}\PY{p}{(}\PY{n}{img}\PY{p}{)}\PY{p}{:}
             \PY{n}{patch} \PY{o}{=} \PY{n}{cv2}\PY{o}{.}\PY{n}{imread}\PY{p}{(}\PY{n}{img}\PY{p}{,} \PY{l+m+mi}{0}\PY{p}{)}
             \PY{n}{patch} \PY{o}{=} \PY{n}{patch}\PY{o}{/}\PY{n}{patch}\PY{o}{.}\PY{n}{max}\PY{p}{(}\PY{p}{)}
             \PY{n}{width}\PY{p}{,} \PY{n}{height} \PY{o}{=} \PY{n}{patch}\PY{o}{.}\PY{n}{shape}
             \PY{n}{k\PYZus{}size} \PY{o}{=} \PY{n}{np}\PY{o}{.}\PY{n}{amax}\PY{p}{(}\PY{n}{patch}\PY{o}{.}\PY{n}{shape}\PY{p}{)}
             \PY{n}{ret} \PY{o}{=} \PY{n}{np}\PY{o}{.}\PY{n}{ones}\PY{p}{(}\PY{p}{(}\PY{n}{k\PYZus{}size}\PY{p}{,} \PY{n}{k\PYZus{}size}\PY{p}{)}\PY{p}{)}
             \PY{n}{x\PYZus{}diff} \PY{o}{=} \PY{n}{math}\PY{o}{.}\PY{n}{floor}\PY{p}{(}\PY{p}{(}\PY{n}{k\PYZus{}size} \PY{o}{\PYZhy{}} \PY{n}{width}\PY{p}{)}\PY{o}{/}\PY{l+m+mi}{2}\PY{p}{)}
             \PY{n}{y\PYZus{}diff} \PY{o}{=} \PY{n}{math}\PY{o}{.}\PY{n}{floor}\PY{p}{(}\PY{p}{(}\PY{n}{k\PYZus{}size} \PY{o}{\PYZhy{}} \PY{n}{height}\PY{p}{)}\PY{o}{/}\PY{l+m+mi}{2}\PY{p}{)}
             \PY{n}{ret}\PY{p}{[}\PY{n}{x\PYZus{}diff}\PY{p}{:}\PY{n}{k\PYZus{}size}\PY{o}{\PYZhy{}}\PY{n}{x\PYZus{}diff}\PY{p}{,}\PY{n}{y\PYZus{}diff}\PY{p}{:}\PY{n}{k\PYZus{}size}\PY{o}{\PYZhy{}}\PY{n}{y\PYZus{}diff}\PY{p}{]} \PY{o}{=} \PY{n}{patch}
             \PY{k}{return} \PY{n}{ret}
         
         \PY{k}{def} \PY{n+nf}{threshold}\PY{p}{(}\PY{n}{supp}\PY{p}{,} \PY{n}{t2}\PY{p}{)}\PY{p}{:}
             \PY{n}{ret} \PY{o}{=} \PY{n}{np}\PY{o}{.}\PY{n}{zeros\PYZus{}like}\PY{p}{(}\PY{n}{supp}\PY{p}{)}
             
             \PY{n}{width}\PY{p}{,} \PY{n}{height}\PY{p}{,} \PY{n}{channels} \PY{o}{=} \PY{n}{supp}\PY{o}{.}\PY{n}{shape}
             \PY{n}{k\PYZus{}width}\PY{p}{,} \PY{n}{k\PYZus{}height} \PY{o}{=} \PY{n}{kernel}\PY{o}{.}\PY{n}{shape}
             \PY{n}{k\PYZus{}half\PYZus{}width} \PY{o}{=} \PY{n}{math}\PY{o}{.}\PY{n}{ceil}\PY{p}{(}\PY{n}{k\PYZus{}width}\PY{o}{/}\PY{l+m+mi}{2}\PY{p}{)}
             \PY{n}{k\PYZus{}half\PYZus{}height} \PY{o}{=} \PY{n}{math}\PY{o}{.}\PY{n}{ceil}\PY{p}{(}\PY{n}{k\PYZus{}height}\PY{o}{/}\PY{l+m+mi}{2}\PY{p}{)}
             
             \PY{k}{for} \PY{n}{x} \PY{o+ow}{in} \PY{n+nb}{range}\PY{p}{(}\PY{n}{k\PYZus{}half\PYZus{}width}\PY{p}{,}\PY{n}{width}\PY{o}{\PYZhy{}}\PY{n}{k\PYZus{}half\PYZus{}width}\PY{p}{)}\PY{p}{:}
                 \PY{k}{for} \PY{n}{y} \PY{o+ow}{in} \PY{n+nb}{range}\PY{p}{(}\PY{n}{k\PYZus{}half\PYZus{}height}\PY{p}{,}\PY{n}{height}\PY{o}{\PYZhy{}}\PY{n}{k\PYZus{}half\PYZus{}height}\PY{p}{)}\PY{p}{:}
                         \PY{n}{theta} \PY{o}{=} \PY{n}{np}\PY{o}{.}\PY{n}{sum}\PY{p}{(}\PY{n}{supp}\PY{p}{[}\PY{n}{x}\PY{p}{,} \PY{n}{y}\PY{p}{,} \PY{p}{:}\PY{p}{]}\PY{p}{)}
                         \PY{k}{if} \PY{n}{theta} \PY{o}{\PYZlt{}} \PY{n}{t2}\PY{p}{:}
                             \PY{n}{ret}\PY{p}{[}\PY{n}{x}\PY{p}{,} \PY{n}{y}\PY{p}{,} \PY{p}{:}\PY{p}{]} \PY{o}{=} \PY{l+m+mf}{1.}
                         \PY{k}{else}\PY{p}{:}
                             \PY{n}{ret}\PY{p}{[}\PY{n}{x}\PY{p}{,} \PY{n}{y}\PY{p}{,} \PY{p}{:}\PY{p}{]} \PY{o}{=} \PY{l+m+mf}{0.}
             \PY{k}{return} \PY{n}{ret}
         
         \PY{c+c1}{\PYZsh{} draw rectangles}
         \PY{k}{def} \PY{n+nf}{plot\PYZus{}squares}\PY{p}{(}\PY{n}{img}\PY{p}{,} \PY{n}{heatmap}\PY{p}{)}\PY{p}{:}
             \PY{n}{p\PYZus{}width}\PY{p}{,} \PY{n}{p\PYZus{}height}\PY{p}{,} \PY{n}{channels} \PY{o}{=} \PY{n}{heatmap}\PY{o}{.}\PY{n}{shape}
             \PY{n}{x\PYZus{}offset} \PY{o}{=} \PY{n}{math}\PY{o}{.}\PY{n}{floor}\PY{p}{(}\PY{n}{kernel}\PY{o}{.}\PY{n}{shape}\PY{p}{[}\PY{l+m+mi}{0}\PY{p}{]}\PY{o}{/}\PY{l+m+mi}{2}\PY{p}{)}
             \PY{n}{y\PYZus{}offset} \PY{o}{=} \PY{n}{math}\PY{o}{.}\PY{n}{floor}\PY{p}{(}\PY{n}{kernel}\PY{o}{.}\PY{n}{shape}\PY{p}{[}\PY{l+m+mi}{1}\PY{p}{]}\PY{o}{/}\PY{l+m+mi}{2}\PY{p}{)}
             \PY{n}{fig}\PY{p}{,}\PY{n}{ax} \PY{o}{=} \PY{n}{plt}\PY{o}{.}\PY{n}{subplots}\PY{p}{(}\PY{l+m+mi}{1}\PY{p}{)}
             \PY{n}{ax}\PY{o}{.}\PY{n}{imshow}\PY{p}{(}\PY{n}{img}\PY{p}{)}
             \PY{k}{for} \PY{n}{x} \PY{o+ow}{in} \PY{n+nb}{range}\PY{p}{(}\PY{n}{p\PYZus{}width}\PY{p}{)}\PY{p}{:}
                 \PY{k}{for} \PY{n}{y} \PY{o+ow}{in} \PY{n+nb}{range}\PY{p}{(}\PY{n}{p\PYZus{}height}\PY{p}{)}\PY{p}{:}
                     \PY{k}{if} \PY{n}{heatmap}\PY{p}{[}\PY{n}{x}\PY{p}{,} \PY{n}{y}\PY{p}{,} \PY{l+m+mi}{0}\PY{p}{]} \PY{o}{==} \PY{l+m+mf}{1.}\PY{p}{:}
                         \PY{n}{rect} \PY{o}{=} \PY{n}{patches}\PY{o}{.}\PY{n}{Rectangle}\PY{p}{(}\PY{p}{(}\PY{n}{y} \PY{o}{\PYZhy{}} \PY{n}{y\PYZus{}offset}\PY{p}{,}\PY{n}{x} \PY{o}{\PYZhy{}} \PY{n}{x\PYZus{}offset}\PY{p}{)}\PY{p}{,}\PY{n}{kernel}\PY{o}{.}\PY{n}{shape}\PY{p}{[}\PY{l+m+mi}{0}\PY{p}{]}\PY{p}{,}\PY{n}{kernel}\PY{o}{.}\PY{n}{shape}\PY{p}{[}\PY{l+m+mi}{1}\PY{p}{]}\PY{p}{,}\PY{n}{linewidth}\PY{o}{=}\PY{l+m+mi}{1}\PY{p}{,}\PY{n}{edgecolor}\PY{o}{=}\PY{l+s+s1}{\PYZsq{}}\PY{l+s+s1}{r}\PY{l+s+s1}{\PYZsq{}}\PY{p}{,}\PY{n}{facecolor}\PY{o}{=}\PY{l+s+s1}{\PYZsq{}}\PY{l+s+s1}{none}\PY{l+s+s1}{\PYZsq{}}\PY{p}{)}
                         \PY{n}{ax}\PY{o}{.}\PY{n}{add\PYZus{}patch}\PY{p}{(}\PY{n}{rect}\PY{p}{)}
             \PY{n}{plt}\PY{o}{.}\PY{n}{show}\PY{p}{(}\PY{p}{)}
             \PY{c+c1}{\PYZsh{}fig.savefig(\PYZdq{}./output.png\PYZdq{}, dpi=240)  \PYZsh{} save the image to the working directory \PYZhy{} for Exercise 4}
\end{Verbatim}


    The entirety of Example 3 is run here to ensure that everything works
after moving the code over. As expected, all 38 detections are present,
with no false positives.

    \begin{Verbatim}[commandchars=\\\{\}]
{\color{incolor}In [{\color{incolor}2}]:} \PY{n}{img} \PY{o}{=} \PY{n}{cv2}\PY{o}{.}\PY{n}{imread}\PY{p}{(}\PY{l+s+s1}{\PYZsq{}}\PY{l+s+s1}{./characters.png}\PY{l+s+s1}{\PYZsq{}}\PY{p}{,} \PY{l+m+mi}{1}\PY{p}{)} \PY{c+c1}{\PYZsh{}load all three channels, due to rendering problems}
        \PY{k}{if} \PY{n}{img} \PY{o+ow}{is} \PY{o+ow}{not} \PY{k+kc}{None}\PY{p}{:}
            \PY{n}{b}\PY{p}{,}\PY{n}{g}\PY{p}{,}\PY{n}{r} \PY{o}{=} \PY{n}{cv2}\PY{o}{.}\PY{n}{split}\PY{p}{(}\PY{n}{img}\PY{p}{)}
            \PY{n}{img} \PY{o}{=} \PY{n}{cv2}\PY{o}{.}\PY{n}{merge}\PY{p}{(}\PY{p}{(}\PY{n}{r}\PY{p}{,}\PY{n}{g}\PY{p}{,}\PY{n}{b}\PY{p}{)}\PY{p}{)}
        \PY{n}{img} \PY{o}{=} \PY{n}{img}\PY{o}{/}\PY{n}{img}\PY{o}{.}\PY{n}{max}\PY{p}{(}\PY{p}{)}
        
        \PY{n}{kernel} \PY{o}{=} \PY{n}{create\PYZus{}kernel}\PY{p}{(}\PY{l+s+s1}{\PYZsq{}}\PY{l+s+s1}{./h.png}\PY{l+s+s1}{\PYZsq{}}\PY{p}{)}
        \PY{n}{probability} \PY{o}{=} \PY{n}{correlation}\PY{p}{(}\PY{n}{img}\PY{p}{,} \PY{n}{kernel}\PY{p}{)}
        \PY{n}{heatmap} \PY{o}{=} \PY{n}{threshold}\PY{p}{(}\PY{n}{probability}\PY{p}{,} \PY{l+m+mf}{0.075} \PY{o}{*} \PY{l+m+mi}{3}\PY{p}{)}
        \PY{n}{plot\PYZus{}squares}\PY{p}{(}\PY{n}{img}\PY{p}{,} \PY{n}{heatmap}\PY{p}{)}
\end{Verbatim}


    \begin{center}
    \adjustimage{max size={0.9\linewidth}{0.9\paperheight}}{output_4_0.png}
    \end{center}
    { \hspace*{\fill} \\}
    
    The function \texttt{roc\_data} is used the FPR and TPR, as discussed
earlier. These will be used to plot the ROC curves.

    To determine the false positives (fp), false negatives (fn), true
positives (tp), and true negatives (tn), the predicted and real values
using the detection matrix from Exerise 3 are used as the baseline. If
both values are the same, it is classified as a true positive or true
negative respectively. If the values differ, it is classified as a false
positive or false negative.

    \begin{Verbatim}[commandchars=\\\{\}]
{\color{incolor}In [{\color{incolor}17}]:} \PY{k}{def} \PY{n+nf}{roc\PYZus{}data}\PY{p}{(}\PY{n}{heatmap}\PY{p}{,} \PY{n}{test}\PY{p}{)}\PY{p}{:}
             \PY{n}{h\PYZus{}width}\PY{p}{,} \PY{n}{h\PYZus{}height}\PY{p}{,} \PY{n}{h\PYZus{}channels} \PY{o}{=} \PY{n}{heatmap}\PY{o}{.}\PY{n}{shape}
             \PY{n}{fp} \PY{o}{=} \PY{l+m+mi}{0}
             \PY{n}{fn} \PY{o}{=} \PY{l+m+mi}{0}
             \PY{n}{tp} \PY{o}{=} \PY{l+m+mi}{0}
             \PY{n}{tn} \PY{o}{=} \PY{l+m+mi}{0}
             \PY{k}{for} \PY{n}{w} \PY{o+ow}{in} \PY{n+nb}{range}\PY{p}{(}\PY{n}{h\PYZus{}width}\PY{p}{)}\PY{p}{:}
                 \PY{k}{for} \PY{n}{h} \PY{o+ow}{in} \PY{n+nb}{range}\PY{p}{(}\PY{n}{h\PYZus{}height}\PY{p}{)}\PY{p}{:}
                     \PY{n}{real} \PY{o}{=} \PY{n}{heatmap}\PY{p}{[}\PY{n}{w}\PY{p}{,} \PY{n}{h}\PY{p}{,} \PY{l+m+mi}{0}\PY{p}{]}
                     \PY{n}{pred} \PY{o}{=} \PY{n}{test}\PY{p}{[}\PY{n}{w}\PY{p}{,} \PY{n}{h}\PY{p}{,} \PY{l+m+mi}{0}\PY{p}{]}
                     \PY{k}{if} \PY{n}{real} \PY{o}{==} \PY{n}{pred}\PY{p}{:}
                         \PY{k}{if} \PY{n}{pred} \PY{o}{\PYZgt{}} \PY{l+m+mi}{0}\PY{p}{:}
                             \PY{n}{tp} \PY{o}{+}\PY{o}{=} \PY{l+m+mi}{1}
                         \PY{k}{else}\PY{p}{:}
                             \PY{n}{tn} \PY{o}{+}\PY{o}{=} \PY{l+m+mi}{1}
                     \PY{k}{else}\PY{p}{:}
                         \PY{k}{if} \PY{n}{pred} \PY{o}{\PYZgt{}} \PY{l+m+mi}{0}\PY{p}{:}
                             \PY{n}{fp} \PY{o}{+}\PY{o}{=} \PY{l+m+mi}{1}
                         \PY{k}{else}\PY{p}{:}
                             \PY{n}{fn} \PY{o}{+}\PY{o}{=} \PY{l+m+mi}{1}
             \PY{c+c1}{\PYZsh{}38 0 0 394762}
             \PY{c+c1}{\PYZsh{}print(fp, fn, tp, tn)}
             \PY{n}{fpr} \PY{o}{=} \PY{n}{fp}\PY{o}{/}\PY{p}{(}\PY{n}{fp} \PY{o}{+} \PY{n}{tn}\PY{p}{)} 
             \PY{n}{tpr} \PY{o}{=} \PY{n}{tp}\PY{o}{/}\PY{p}{(}\PY{n}{tp} \PY{o}{+} \PY{n}{fn}\PY{p}{)}
             \PY{k}{return} \PY{n}{fpr}\PY{p}{,} \PY{n}{tpr}
\end{Verbatim}


    Using the \texttt{roc\_data} function, the function \texttt{create\_ROC}
loops through different detection thresholds and plot the FPR and TPR
respectively. The TPR is represented on the Y axis and FPR on the X
axis.

    \begin{Verbatim}[commandchars=\\\{\}]
{\color{incolor}In [{\color{incolor}4}]:} \PY{k}{def} \PY{n+nf}{create\PYZus{}ROC}\PY{p}{(}\PY{n}{heatmap}\PY{p}{,} \PY{n}{probability\PYZus{}test}\PY{p}{)}\PY{p}{:}
            \PY{n}{x\PYZus{}list} \PY{o}{=} \PY{p}{[}\PY{p}{]}
            \PY{n}{y\PYZus{}list} \PY{o}{=} \PY{p}{[}\PY{p}{]}
            \PY{k}{for} \PY{n}{theta} \PY{o+ow}{in} \PY{n}{np}\PY{o}{.}\PY{n}{arange}\PY{p}{(}\PY{l+m+mf}{0.05}\PY{p}{,} \PY{l+m+mf}{0.9}\PY{p}{,} \PY{l+m+mf}{0.05}\PY{p}{)}\PY{p}{:}
                \PY{n}{heatmap\PYZus{}test} \PY{o}{=} \PY{n}{threshold}\PY{p}{(}\PY{n}{probability\PYZus{}test}\PY{p}{,} \PY{n}{theta} \PY{o}{*} \PY{l+m+mi}{3}\PY{p}{)}
                \PY{n}{fpr}\PY{p}{,} \PY{n}{tpr} \PY{o}{=} \PY{n}{roc\PYZus{}data}\PY{p}{(}\PY{n}{heatmap}\PY{p}{,} \PY{n}{heatmap\PYZus{}test}\PY{p}{)}
                \PY{n}{x\PYZus{}list}\PY{o}{.}\PY{n}{append}\PY{p}{(}\PY{n}{fpr}\PY{p}{)}
                \PY{n}{y\PYZus{}list}\PY{o}{.}\PY{n}{append}\PY{p}{(}\PY{n}{tpr}\PY{p}{)}
            \PY{n}{plt}\PY{o}{.}\PY{n}{scatter}\PY{p}{(}\PY{n}{x\PYZus{}list}\PY{p}{,} \PY{n}{y\PYZus{}list}\PY{p}{)}
            \PY{n}{plt}\PY{o}{.}\PY{n}{xlim}\PY{p}{(} \PY{o}{\PYZhy{}}\PY{l+m+mf}{0.1}\PY{p}{,} \PY{l+m+mf}{1.1} \PY{p}{)}
            \PY{n}{plt}\PY{o}{.}\PY{n}{ylim}\PY{p}{(} \PY{o}{\PYZhy{}}\PY{l+m+mf}{0.1}\PY{p}{,} \PY{l+m+mf}{1.1} \PY{p}{)}
            \PY{n}{plt}\PY{o}{.}\PY{n}{show}\PY{p}{(}\PY{p}{)}
\end{Verbatim}


    As we can see, the ``perfect'' threshold that was found in Exercise 3 is
following the trend of ``ideal behavior'', according to the slides.
There are no false positives or false negatives.

    \begin{Verbatim}[commandchars=\\\{\}]
{\color{incolor}In [{\color{incolor}5}]:} \PY{n}{create\PYZus{}ROC}\PY{p}{(}\PY{n}{heatmap}\PY{p}{,} \PY{n}{probability}\PY{p}{)}
\end{Verbatim}


    \begin{center}
    \adjustimage{max size={0.9\linewidth}{0.9\paperheight}}{output_11_0.png}
    \end{center}
    { \hspace*{\fill} \\}
    
    The next step is to introduce noise into the situation to see how it
alters the ROC curve. The goal is to increase FP and FN rates. To aid in
runtime and simplicity, gray Gaussian noise is added to the image. RGB
noise requires an extra three iterations per pixel to check, and would
make the runtime very long.

    \begin{Verbatim}[commandchars=\\\{\}]
{\color{incolor}In [{\color{incolor}18}]:} \PY{n}{gray\PYZus{}noise} \PY{o}{=} \PY{n}{np}\PY{o}{.}\PY{n}{random}\PY{o}{.}\PY{n}{normal}\PY{p}{(}\PY{l+m+mi}{0}\PY{p}{,} \PY{l+m+mf}{0.03}\PY{p}{,} \PY{p}{(}\PY{n}{img}\PY{o}{.}\PY{n}{shape}\PY{p}{[}\PY{l+m+mi}{0}\PY{p}{]}\PY{p}{,} \PY{n}{img}\PY{o}{.}\PY{n}{shape}\PY{p}{[}\PY{l+m+mi}{1}\PY{p}{]}\PY{p}{)}\PY{p}{)} \PY{c+c1}{\PYZsh{}mean = 0, std = 0.03}
         \PY{n}{rgb\PYZus{}noise} \PY{o}{=} \PY{n}{np}\PY{o}{.}\PY{n}{asarray}\PY{p}{(}\PY{n}{np}\PY{o}{.}\PY{n}{dstack}\PY{p}{(}\PY{p}{(}\PY{n}{gray\PYZus{}noise}\PY{p}{,} \PY{n}{gray\PYZus{}noise}\PY{p}{,} \PY{n}{gray\PYZus{}noise}\PY{p}{)}\PY{p}{)}\PY{p}{)}
         \PY{n}{img\PYZus{}noise} \PY{o}{=} \PY{n}{img} \PY{o}{+} \PY{n}{rgb\PYZus{}noise}
         \PY{n}{img\PYZus{}noise} \PY{o}{=} \PY{n}{np}\PY{o}{.}\PY{n}{absolute}\PY{p}{(}\PY{n}{img\PYZus{}noise}\PY{o}{/}\PY{n}{img\PYZus{}noise}\PY{o}{.}\PY{n}{max}\PY{p}{(}\PY{p}{)}\PY{p}{)}
         \PY{n}{plt}\PY{o}{.}\PY{n}{imshow}\PY{p}{(}\PY{n}{img\PYZus{}noise}\PY{p}{)}
         \PY{n}{plt}\PY{o}{.}\PY{n}{show}\PY{p}{(}\PY{p}{)}
\end{Verbatim}


    \begin{center}
    \adjustimage{max size={0.9\linewidth}{0.9\paperheight}}{output_13_0.png}
    \end{center}
    { \hspace*{\fill} \\}
    
    As we can see the number of detections has been lowered to 16. There are
no false negatives with the given threshold, but we can determine if
some exist given the ROC curve.

    \begin{Verbatim}[commandchars=\\\{\}]
{\color{incolor}In [{\color{incolor}7}]:} \PY{n}{probability\PYZus{}noise} \PY{o}{=} \PY{n}{correlation}\PY{p}{(}\PY{n}{img\PYZus{}noise}\PY{p}{,} \PY{n}{kernel}\PY{p}{)}
        \PY{n}{heatmap\PYZus{}noise} \PY{o}{=} \PY{n}{threshold}\PY{p}{(}\PY{n}{probability\PYZus{}noise}\PY{p}{,} \PY{l+m+mf}{0.075} \PY{o}{*} \PY{l+m+mi}{3}\PY{p}{)}
        \PY{n}{plot\PYZus{}squares}\PY{p}{(}\PY{n}{img\PYZus{}noise}\PY{p}{,} \PY{n}{heatmap\PYZus{}noise}\PY{p}{)}
\end{Verbatim}


    \begin{center}
    \adjustimage{max size={0.9\linewidth}{0.9\paperheight}}{output_15_0.png}
    \end{center}
    { \hspace*{\fill} \\}
    
    As we can see, there is just one case where false positives occur. False
negatives are more common, however.

    \begin{Verbatim}[commandchars=\\\{\}]
{\color{incolor}In [{\color{incolor}8}]:} \PY{n}{create\PYZus{}ROC}\PY{p}{(}\PY{n}{heatmap}\PY{p}{,} \PY{n}{probability\PYZus{}noise}\PY{p}{)}
\end{Verbatim}


    \begin{center}
    \adjustimage{max size={0.9\linewidth}{0.9\paperheight}}{output_17_0.png}
    \end{center}
    { \hspace*{\fill} \\}
    
    We repeat the test with a higher level of noise.

    \begin{Verbatim}[commandchars=\\\{\}]
{\color{incolor}In [{\color{incolor}9}]:} \PY{n}{gray\PYZus{}noise} \PY{o}{=} \PY{n}{np}\PY{o}{.}\PY{n}{random}\PY{o}{.}\PY{n}{normal}\PY{p}{(}\PY{l+m+mi}{0}\PY{p}{,} \PY{l+m+mf}{0.1}\PY{p}{,} \PY{p}{(}\PY{n}{img}\PY{o}{.}\PY{n}{shape}\PY{p}{[}\PY{l+m+mi}{0}\PY{p}{]}\PY{p}{,} \PY{n}{img}\PY{o}{.}\PY{n}{shape}\PY{p}{[}\PY{l+m+mi}{1}\PY{p}{]}\PY{p}{)}\PY{p}{)}
        \PY{n}{rgb\PYZus{}noise} \PY{o}{=} \PY{n}{np}\PY{o}{.}\PY{n}{asarray}\PY{p}{(}\PY{n}{np}\PY{o}{.}\PY{n}{dstack}\PY{p}{(}\PY{p}{(}\PY{n}{gray\PYZus{}noise}\PY{p}{,} \PY{n}{gray\PYZus{}noise}\PY{p}{,} \PY{n}{gray\PYZus{}noise}\PY{p}{)}\PY{p}{)}\PY{p}{)}
        \PY{n}{img\PYZus{}noise} \PY{o}{=} \PY{n}{img} \PY{o}{+} \PY{n}{rgb\PYZus{}noise}
        \PY{n}{img\PYZus{}noise} \PY{o}{=} \PY{n}{np}\PY{o}{.}\PY{n}{absolute}\PY{p}{(}\PY{n}{img\PYZus{}noise}\PY{o}{/}\PY{n}{img\PYZus{}noise}\PY{o}{.}\PY{n}{max}\PY{p}{(}\PY{p}{)}\PY{p}{)}
        \PY{n}{plt}\PY{o}{.}\PY{n}{imshow}\PY{p}{(}\PY{n}{img\PYZus{}noise}\PY{p}{)}
        \PY{n}{plt}\PY{o}{.}\PY{n}{show}\PY{p}{(}\PY{p}{)}
\end{Verbatim}


    \begin{center}
    \adjustimage{max size={0.9\linewidth}{0.9\paperheight}}{output_19_0.png}
    \end{center}
    { \hspace*{\fill} \\}
    
    This time, we can visibly see that the curve is no longer at the
``ideal'', which indicates lower detector performance. It appears to be
directly related to amount of noise, as expected.

    \begin{Verbatim}[commandchars=\\\{\}]
{\color{incolor}In [{\color{incolor}10}]:} \PY{n}{probability\PYZus{}noise} \PY{o}{=} \PY{n}{correlation}\PY{p}{(}\PY{n}{img\PYZus{}noise}\PY{p}{,} \PY{n}{kernel}\PY{p}{)}
         \PY{n}{create\PYZus{}ROC}\PY{p}{(}\PY{n}{heatmap}\PY{p}{,} \PY{n}{probability\PYZus{}noise}\PY{p}{)}
\end{Verbatim}


    \begin{center}
    \adjustimage{max size={0.9\linewidth}{0.9\paperheight}}{output_21_0.png}
    \end{center}
    { \hspace*{\fill} \\}
    
    Let us try an example with a lot of noise. We should expect to see
greatly reduced detector performance.

    \begin{Verbatim}[commandchars=\\\{\}]
{\color{incolor}In [{\color{incolor}13}]:} \PY{n}{gray\PYZus{}noise} \PY{o}{=} \PY{n}{np}\PY{o}{.}\PY{n}{random}\PY{o}{.}\PY{n}{normal}\PY{p}{(}\PY{l+m+mi}{0}\PY{p}{,} \PY{l+m+mf}{0.3}\PY{p}{,} \PY{p}{(}\PY{n}{img}\PY{o}{.}\PY{n}{shape}\PY{p}{[}\PY{l+m+mi}{0}\PY{p}{]}\PY{p}{,} \PY{n}{img}\PY{o}{.}\PY{n}{shape}\PY{p}{[}\PY{l+m+mi}{1}\PY{p}{]}\PY{p}{)}\PY{p}{)}
         \PY{n}{rgb\PYZus{}noise} \PY{o}{=} \PY{n}{np}\PY{o}{.}\PY{n}{asarray}\PY{p}{(}\PY{n}{np}\PY{o}{.}\PY{n}{dstack}\PY{p}{(}\PY{p}{(}\PY{n}{gray\PYZus{}noise}\PY{p}{,} \PY{n}{gray\PYZus{}noise}\PY{p}{,} \PY{n}{gray\PYZus{}noise}\PY{p}{)}\PY{p}{)}\PY{p}{)}
         \PY{n}{img\PYZus{}noise} \PY{o}{=} \PY{n}{img} \PY{o}{+} \PY{n}{rgb\PYZus{}noise}
         \PY{n}{img\PYZus{}noise} \PY{o}{=} \PY{n}{np}\PY{o}{.}\PY{n}{absolute}\PY{p}{(}\PY{n}{img\PYZus{}noise}\PY{o}{/}\PY{n}{img\PYZus{}noise}\PY{o}{.}\PY{n}{max}\PY{p}{(}\PY{p}{)}\PY{p}{)}
         \PY{n}{plt}\PY{o}{.}\PY{n}{imshow}\PY{p}{(}\PY{n}{img\PYZus{}noise}\PY{p}{)}
         \PY{n}{plt}\PY{o}{.}\PY{n}{show}\PY{p}{(}\PY{p}{)}
\end{Verbatim}


    \begin{center}
    \adjustimage{max size={0.9\linewidth}{0.9\paperheight}}{output_23_0.png}
    \end{center}
    { \hspace*{\fill} \\}
    
    As expected, detector performance has been greatly diminished, and the
effects are very noticable on the ROC curve.

    \begin{Verbatim}[commandchars=\\\{\}]
{\color{incolor}In [{\color{incolor}14}]:} \PY{n}{probability\PYZus{}noise} \PY{o}{=} \PY{n}{correlation}\PY{p}{(}\PY{n}{img\PYZus{}noise}\PY{p}{,} \PY{n}{kernel}\PY{p}{)}
         \PY{n}{create\PYZus{}ROC}\PY{p}{(}\PY{n}{heatmap}\PY{p}{,} \PY{n}{probability\PYZus{}noise}\PY{p}{)}
\end{Verbatim}


    \begin{center}
    \adjustimage{max size={0.9\linewidth}{0.9\paperheight}}{output_25_0.png}
    \end{center}
    { \hspace*{\fill} \\}
    

    % Add a bibliography block to the postdoc
    
    
    
    \end{document}
